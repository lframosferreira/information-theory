\documentclass{article}
\usepackage{graphicx}
\usepackage{float}
\usepackage{subcaption}
\usepackage{amsthm}
\usepackage[colorlinks=true, allcolors=blue]{hyperref}

\bibliographystyle{alpha}

\title{Information Theory \\ \large Problem set 02 - Probability, entropy and inference}
\author{Luís Felipe Ramos Ferreira}
\date{\href{mailto:lframos\_ferreira@outlook.com}{\texttt{lframos\_ferreira@outlook.com}}
}

\begin{document}

\maketitle

\begin{enumerate}
  \item \begin{enumerate}
          \item The frequentist interpretation of probability is, obviously, associated with the concept of frequency. It states that the probability \(p(s)\) of a event \(s\) should reflect the frequency that the event \(s\) happens compared to the rest of the events in the sample space, as the number of experiments goes to infinity. On the other hand, the bayesian interpretation of probability is more subjective, as it assumes the probability \(p(s)\) of a event \(s\) is the degree of belief we should have that the outcome of an experiment in the sample space will be \(s\).
      \item b
      \item c
  \end{enumerate}

\item oi
\end{enumerate}

\end{document}
