\documentclass{article}
\usepackage{graphicx}
\usepackage{float}
\usepackage{subcaption}
\usepackage{amsmath}
\usepackage[colorlinks=true, allcolors=blue]{hyperref}
\bibliographystyle{unsrtnat}
\usepackage[square, numbers]{natbib}

\bibliographystyle{alpha}

\title{Information Theory \\ \large Problem Set 04 - Symbol Codes}
\author{Luís Felipe Ramos Ferreira}
\date{\href{mailto:lframos\_ferreira@outlook.com}{\texttt{lframos\_ferreira@outlook.com}}
}

\begin{document}

\maketitle

\begin{enumerate}
	\item \begin{enumerate}
		      \item A (binary) symbol code for an ensemble, denoted by \(C\), is a function that maps the outcomes of the ensemble to a set os binary strings. In particular, this set of strings is a subset of \(\{0, 1\}^+\), which denotes the set of all binary strings of non zero length. The extended code for the ensemble, denoted by \(C^+\), is a function from \(\mathcal{A}_X^+\) to \(\{0, 1\}^+\). More precisely, it represents the concatenation of the codewords of a ordered set of outcomes from the ensemble.
		      \item A symbol code is uniquely decodeable when no element is mapped to the same codeword. It is easy to see that is true based on the pidgeonhole principle. More formally, a code \(C(x)\) is uniquely decodeable if, under the extended code \(C^+\), we have:
		            \[\forall x, y \in \mathcal{A}_X^+, x \neq y \Rightarrow c^+(x) \neq c^+(y)\]

                    A symbol code is prefix-free if no codeword is a prefix of any other codeword, as stated by McKay \cite{MacKay}.
		      \item c
		      \item d
	      \end{enumerate}

	\item 2
	\item 3
	\item \begin{enumerate}

	      \end{enumerate}
	\item 5
	\item 6
\end{enumerate}

\bibliography{sample}

\end{document}
