\documentclass{article}
\usepackage{graphicx}
\usepackage{float}
\usepackage{subcaption}
\usepackage{listings}
\usepackage{amsmath}
\usepackage{amsfonts}
\usepackage[square, numbers]{natbib}
\bibliographystyle{unsrtnat}
\usepackage[colorlinks=true, allcolors=blue]{hyperref}
\usepackage{xcolor}
\bibliographystyle{alpha}

\title{Information Theory \\ \large Problem Set 10 - Advanced Information Measures}
\author{Luís Felipe Ramos Ferreira}
\date{\href{mailto:lframos\_ferreira@outlook.com}{\texttt{lframos\_ferreira@outlook.com}}
}

\begin{document}

\maketitle

\begin{enumerate}
	\item \begin{enumerate}
		      \item Given a set of secret values, we can define the set of all posible probability distributions over this set of secret values. A state of knowledge of a agent about
		            the secret (or state of the world) can be defined as one of such probability distributions that the set of posible secrets can have.

		      \item A information measure is a mapping from a state of knowledge to a real number, i.e, it's a function \(f: \mathcal{X} \mathbb{D} \rightarrow \mathbb{R}\), where \(\mathca{X} \mathbb{D}\)
		            is a set of probability distributions about a set of secrets \(\mathcal{X}\), \(f\) is the function itself and \(\mathbb{R}\) is a real number that represents how much knowledge the agent had about the secret
		            before knowing it's value.
		      \item The mathematical definiton of information measure is a function that maps every posible input of possible states of knowledge to a real number that represents the value of that knowledge. The operational significance
		            of information measure it's the meaning of the real number returned by the function mentioned, i.e., what that value means and how can we interpretate it.
	      \end{enumerate}
	\item 2
\end{enumerate}

\bibliography{sample}
\nocite{*}

\end{document}
