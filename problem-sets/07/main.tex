\documentclass{article}
\usepackage{graphicx}
\usepackage{float}
\usepackage{subcaption}
\usepackage{amsmath}
\usepackage[square, numbers]{natbib}
\bibliographystyle{unsrtnat}
\usepackage[colorlinks=true, allcolors=blue]{hyperref}

\bibliographystyle{alpha}

\title{Information Theory \\ \large Problem Set 07 - Communication Over a Noisy Channel}
\author{Luís Felipe Ramos Ferreira}
\date{\href{mailto:lframos\_ferreira@outlook.com}{\texttt{lframos\_ferreira@outlook.com}}
}

\begin{document}

\maketitle

\begin{enumerate}
    \item  \begin{enumerate}
            \item A discrete memoryless channel is characterized in a formal way as an input alphabet \(\mathcal{A}_x\), an output alphabet \(\mathcal{A}_y\) and a set of conditional probability distributions \(P(y | x)\), for each \(x \in \mathcal{A}_x\).
            \item If a channel is noisy, there is a non null probability that the message sent by a source will be different than the message received by the receiver. The noisy channel will affect the message, making it not easy to decode. To solve this issue, some kind of approach needs to be used to establish a reliable communication over a noisy channel.
            \item 
        \end{enumerate}
    \item 2
    \item 3
    \item 4
    \item 5

\end{enumerate}

\bibliography{sample}

\end{document}
