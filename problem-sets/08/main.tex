\documentclass{article}
\usepackage{graphicx}
\usepackage{float}
\usepackage{subcaption}
\usepackage{listings}
\usepackage{amsmath}
\usepackage[square, numbers]{natbib}
\bibliographystyle{unsrtnat}
\usepackage[colorlinks=true, allcolors=blue]{hyperref}
\usepackage{xcolor}

\definecolor{mGreen}{rgb}{0,0.6,0}
\definecolor{mGray}{rgb}{0.5,0.5,0.5}
\definecolor{mPurple}{rgb}{0.58,0,0.82}
\definecolor{backgroundColour}{rgb}{0.95,0.95,0.92}

\lstdefinestyle{CStyle}{
    backgroundcolor=\color{backgroundColour},   
    commentstyle=\color{mGreen},
    keywordstyle=\color{magenta},
    numberstyle=\tiny\color{mGray},
    stringstyle=\color{mPurple},
    basicstyle=\footnotesize,
    breakatwhitespace=false,         
    breaklines=true,                 
    captionpos=b,                    
    keepspaces=true,                 
    numbers=left,                    
    numbersep=5pt,                  
    showspaces=false,                
    showstringspaces=false,
    showtabs=false,                  
    tabsize=2,
    language=C
}
\bibliographystyle{alpha}

\title{Information Theory \\ \large Problem Set 08 - Kolmogorov Complexity and Universal Probability}
\author{Luís Felipe Ramos Ferreira}
\date{\href{mailto:lframos\_ferreira@outlook.com}{\texttt{lframos\_ferreira@outlook.com}}
}

\begin{document}

\maketitle

\begin{enumerate}
	\item \begin{enumerate}
		      \item The Kolgomorov complexity of a string  \(s\) is the length of the shortest program that can be passed to a UTM (Universal Turing Machine),
		            so that the UTM outputs the string \(s\) and then halts.
		      \item A string is considered truly random when it's Kolgomorov complexity is bigger tah or equal to it's own length.
		            A string that is truly random, for example, is the result string of subsequent flip of a coin. Since the flip of the coin is random, there is no way we can compress it using some kind of algorithm. A string that looks random but is not is the decimal expansion of the irrational number
		            Pi.
		      \item The universal probability of a string \(s\) is the probability that, when we give a random program to a UTM, \(s\) is the output of the execution of such program.
		            It is related to its Kolmogorov complexity by the equation \(P_U(s) \approx 2^{-K(s)}\), where \(P_U(s)\) is the universal probabilty
		            of the string \(s\) and \(K(s)\) is the Kolgomorov complexity of the string \(s\).
	      \end{enumerate}

	\item We can construct a program that outputs the concatenation \(xy\) in the following way. First, use the fact that the program that describes \(x\),
	      which has at most \(K(x)\) bits and the program that describes \(y\) has at most \(y\) bits. Then, create a program that first outputs \(x\), using at most \(K(x)\) bits, and then outputs \(y\), which uses at most \(K(y)\) bits.
	      The piece of code that tells the order of the output is constant size and do not depend on the strings. So, the created program, which outputs the concatenation \(xy\), has a Kolgomorov complexity of at most \(K(x) + K(y) + c\).
	\item \begin{enumerate}
		      \item We can construct a program that first describes the number \(n_1\) using at most \(K(n_1)\) bits, then uses the language of the program that represents the sum operator (for example, the operator '+'), and then
		            uses the string that represents the number \(n_2\) using at most \(K(n_2)\) bits. Since the sum operator described have a constant size, the description we showed has at most \(K(n_1) + K(n_2) + c\) bits and describes number
		            that is the sum of numbers \(n_1\) and \(n_2\).
		      \item As seen in the classes, strings generated by the flip of a fair coin are complex, since we can't describe them in a algorithmic way. In this scenario, we can flip a fair coint \(n\) times and create two strings based
		            on the values of the flip. If the flip is heads, we put a \(1\) in the string \(n_1\) and a \(0\) in the string \(n_2\). Otherwise, we do the opposite. Both strings are complex, since they were created by the random flip of a fair coin. But if we add them, considering the binary number they represent, we achieve, by construction,
		            a string of \(N 1's\), which is very simple and can be described very easily.
	      \end{enumerate}
	\item For both questions, all we need is to find the universal probability of the code that prints the desired outputs with their specific prefixes.

	      \begin{enumerate}
		      \item In python, the following code will print a sequence of \(n\) zeros.
		            \begin{lstlisting}[style=Cstyle]
				  	print(n*"0")
				  \end{lstlisting}
		            The code only has \(11\) characters and the cost to represent the number \(n\) in the code is roughly \(log_{10}(n)\) characters, so in total we have \(11 + log_{10}n\) characters, which is approximattely \(88 + 8 * log_{10}n\) bits, if we use ASCII.
		            So, we can infer that the Kolgomorov complexity of the string with a sequence of \(n\) zeros is less than \(88 + 8 * log_{10}n\) and that the universal probability of such string \(s\) is \(2^K(s) \geq 2^{-(88 + log_{10}n)}\).
		      \item All we need to do is find the universal probability of the code that produces the first \(n\) digits of Pi. We know such programs can be written in a language such as \(C\), so we can write a program in \(C\) that returns the first \(n\) digits of Pi using a limited number of bits, using some kind of formula for computing
		            the value of Pi. For example if we use the following code, we would return Pi with the first 7 digits of Pi.
		            \begin{lstlisting}[style=CStyle]
				  #include <stdio.h>
				  int main() {
					  double pi = 0.0;
					  double m = 1.0f;
					  for (int i = 1; i < 10000000; i += 2) {
						  pi += m * (1.0f / i);
						  m *= -1.0f;
					  }
					  printf("%f\n", 4.0f*pi);
					  return 0;
				  }
				  \end{lstlisting}
		            The code above has a size of 4.0K, which is around 32768 bits. So, we know that the Kolgomorov complexity of the first 7 digits of Pi is less than 32768 and that it's universal probability is approximately bigger than \(2^{-32768}\).
		            The problem is more complex for general \(n\), but the result above gives a rough idea of how to solve the problem.
	      \end{enumerate}
\end{enumerate}

\bibliography{sample}
\nocite{*}

\end{document}
