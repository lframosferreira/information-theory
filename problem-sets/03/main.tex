\documentclass{article}
\usepackage{graphicx}
\usepackage{float}
\usepackage{subcaption}
\usepackage{amsmath}
\usepackage[square, numbers]{natbib}
\bibliographystyle{unsrtnat}
\usepackage[colorlinks=true, allcolors=blue]{hyperref}

\bibliographystyle{alpha}

\title{Information Theory \\ \large Problem Set 03 - The Source Coding Theorem}
\author{Luís Felipe Ramos Ferreira}
\date{\href{mailto:lframos\_ferreira@outlook.com}{\texttt{lframos\_ferreira@outlook.com}}
}

\begin{document}

\maketitle

\begin{enumerate}

	\item \begin{enumerate}
		      \item The Shannon information content \(h(x)\) of the outcome \(x\) of a random experiment is defined as:
		            \[ h(x) = log_2 \frac{1}{p_x}\],
		            where \(p_x\) represents the probability of observing the outcome \(x\). The value \(h(x)\) means, in a general sense of the word, the amount of information we gain about the state of the world after running the experiment and obtaining the result \(x\). We can also say it measures the amount of uncertainty of that outcome. If it's probability \(p_x\) is big, there is no "surprise" when the result \(x\) is obtained, but if \(p_x\) is small, you get very "surprised" with that information.
		      \item The entropy \(H(X)\) of an ensemble \(X\) is defined as:
		            \[H(X) = \sum_{x \in X} p_x log_2 \frac{1}{p_x}\]
		            The value \(H(X)\) is a measure of the average information content of the ensemble \(X\). As the formula states, it is a weighted average of the Shannon information content of each outcome \(x\) of the emsemble, where the weight of each outcome is it's probability.
		      \item A convex function is a function where, for any pair of points in the graph of the function, the line between those points lies above the graph between the points. In more formal therms, as stated by MacKay in \cite{MacKay}, a function \(f\) is convex if for all \(x_1, x_2 \in (a, b)\) and \(0 \leq \lambda \leq 1\),
		            \[f(\lambda x_1 + (1 - \lambda)x_2) \leq \lambda f(x_1) + (1 - \lambda) f(x_2)\]

		            For example:

		            \begin{enumerate}
			            \item \(f(x) = x^2\) is convex over \((-\infty, \infty)\);
			            \item \(f(x) = e^x\) is convex over \((-\infty, \infty)\);
			            \item \(f(x) = sin(x)\) is not convex over \((-\infty, \infty)\);
			            \item \(f(x) = x^3\) is not convex over \((-\infty, \infty)\);
		            \end{enumerate}
		      \item Jensen's inequality states that if \(f\) is a convex function and \(x\) is a random variable then:
		            \[\mathcal{E}[f(x)] \geq f(\mathcal{E}[x])\],
		            where \(\mathcal{E}\) denotes expectation. DISCORRER MAIS SOBRE PARA APRENDER
		      \item The formula for the raw bit content of an ensemble \(X\) is
		            \[H_0(X) = log_2 \mid \mathcal{A}_X \mid\],
		            where \(\mathcal{A}_X\) is the set of possible outcomes of the ensemble \(X\). It represents, in general, the number of binary questions that are needed to identify an outcome \(x\) from \(X\) for sure. It can also be seen as the smallest length necessary to map each outcome of \(X\) to a binary string.
		      \item
	      \end{enumerate}

	\item As stated before, the entropy \(H(X)\) of an ensemble can be defined as:
	      \[H(X) = \sum_{x \in X} p_x log_2 \frac{1}{p_x}\],
	      where \(p_x\) represents the probability of observing the outcome \(x\).
	      Note that, for each outcome \(x\), we have \(p_x \geq 0\), which implies that \(p_x log_2 \frac{1}{p_x} \geq 0\) (For convention, we assume \(0 log_2 \frac{1}{0}\) to be \(0\), since it's where the limit goes. Therefore, for each outcome, there is a contribution to \(H(X)\) that greater or equal to zero. At last, we have the sum over the contribution of every outcome, values that are greater or equal to zero, which is also greater or equal to zero, so \(H(X) \geq 0\).
	\item \begin{enumerate}

		      \item
		            \[\mathcal{E}(f(x)) = p_a f_a + p_b f_b + p_c f_c = 0.1 * 10 + 0.2 * 5 + 0.7 * \frac{10}{7}\]
		            \[\mathcal{E}(f(x)) = 1 + 1 + 1 = 3\]

		            \[\mathcal{E}(\frac{1}{P(x)}) = p_a \frac{1}{p_a} + p_b \frac{1}{p_b} + p_c \frac{1}{p_c} = 1 + 1 + 1 = 3\]

		      \item For an arbitrary ensemble \(X\), \(\mathcal{E}(\frac{1}{P(x)})\) is exactly the number of possible outcomes of \(X\).
		            \[\mathcal{E}(\frac{1}{P(x)}) = \sum_{x \in \mathcal{A}_X} P(x) \frac{1}{P(x)} = \sum_{x \in \mathcal{A}_X} 1 = \mid \mathcal{A}_X \mid\]

		      \item jensen
	      \end{enumerate}

	\item The problem clearly states a geometric distribution, i. e., the probability distirbution of the number of Bernoulli trials needed to get one success. In this scenario, getting heads is considered success. Let's consider \(P_{heads} = P_{tails} = \frac{1}{2}\). For a geometric distribution, we have that the probability of the number of flips required until the first head occurs to be:
	      \[P(X = k) = (1 - P_{tails})^{k-1} P_{heads} = (\frac{1}{2})^k\]
	      Therefore, we have:

	      \[H(X) = \sum_{k=0}^{\infty} P(X = k) log_2 \frac{1}{P(X = k)} = \sum_{k=0}^{\infty} (\frac{1}{2})^k log_2 \frac{1}{(\frac{1}{2})^k}\]
	      \[H(X) = \sum_{k=0}^{\infty} (\frac{1}{2})^k k = \frac{\frac{1}{2}}{(1 - \frac{1}{2})^2} = \frac{\frac{1}{2}}{\frac{1}{4}} = 2\]

	\item Handmande exercise.



	\item No, it couldn't. It's easy to see that based on the pidgeonhle principle. If every possible outcome were compressed into a binary code of length shorter than \(H_0(X)\), then some outcomes would be mapped to the same binary code, which configures a type of lossi compressor. Therefore, it wouldn't be possible to map \(c\) back to \(x\) with \(100\%\) reliability.

	\item

\end{enumerate}

\bibliography{sample}

\end{document}
